
%### Packages und Configuration ######################
\usepackage[left= 3cm, right = 2cm, bottom = 2cm, top = 2cm]{geometry}
%für das Umschmiegen des Textes um ein Bild
\usepackage{wrapfig}
\usepackage[utf8]{inputenc}
\usepackage{xcolor}
\usepackage[ngerman]{babel}

% \usepackage[chapter]{minted}
% \usemintedstyle{solarized-light}
% \setminted{linenos=true}

\usepackage[T1]{fontenc}
\usepackage{graphicx, subfig}
\graphicspath{{img/}}

\usepackage{fancyhdr}
\usepackage{lmodern}
\renewcommand*\familydefault{\sfdefault}

\usepackage{color}
\usepackage{underscore}
\usepackage{acronym}
% zusätzliche Schriftzeichen der American Mathematical Society
\usepackage{amsfonts}
\usepackage{amsmath}


%### Code Listing Style ##############################
\usepackage{listings}
\definecolor{backcolour}{rgb}{0.95,0.95,0.92}              
\definecolor{commentcolor}{rgb}{0.497495, 0.497587, 0.497464}
\definecolor{keywordcolor}{rgb}{0.000000, 0.000000, 0.635294}
\definecolor{ndkeywordcolor}{rgb}{0,0.6,0.2}
\definecolor{numbercolor}{rgb}{0.5,0.5,0.5}
\definecolor{stringcolor}{rgb}{0.6,0.3,0.1}

\definecolor{deepred}{rgb}{0.6,0,0}
\definecolor{purple}{rgb}{0.58,0,0.82}

% Einstellen der Benutzerdefinierten Code-Farben
\lstdefinestyle{mystyle}{
    backgroundcolor=\color{backcolour},   
    commentstyle=\color{commentcolor},
    otherkeywords={self},
    keywordstyle=\color{keywordcolor}\bfseries,
  	ndkeywordstyle=\color{ndkeywordcolor}\bfseries,
    numberstyle=\tiny\color{numbercolor},
    stringstyle=\color{stringcolor}\bfseries,
    emph={import},
    emphstyle=\color{deepred},
    basicstyle=\footnotesize,  
    frame=single,
    breakatwhitespace=false,         
    breaklines=false,                 
    captionpos=b,                    
    keepspaces=true,                 
    numbers=left,                    
    numbersep=5pt,                                        
    showspaces=false,                
    showstringspaces=false,
    showtabs=false,                  
    tabsize=2,
   postbreak=\mbox{\textcolor{deepred}{$\hookrightarrow$}\space}
}
\lstset{style=mystyle}
\renewcommand{\lstlistingname}{Code}

%### Dokument Config #################################
%Schriftgröße der Bildunterschriften
\usepackage[font=small,labelfont=bf]{caption}
%Abstand Caption-Bild
\setlength{\abovecaptionskip}{5pt plus 3pt minus 2pt}

%"Tiefe" für Nummerierung ändern (wie viele Unterkategorien sollen nummeriert werden
\setcounter{secnumdepth}{5}

%### Dokumentinformationen ###########################
\usepackage[
	pdftitle={Todo: Mein DA Titel},
	pdfsubject={},
	pdfauthor={Todo: Author},
	pdfauthor={Todo: Author},
	pdfkeywords={},	
	%Links nicht einrahmen
	hidelinks
]{hyperref}

%### Shortcut Kommando fuer hspace ###################
\newcommand\tab[1][1cm]{\hspace*{#1}}

%wichtig für Abstand oben
%\renewcommand*{\chapterheadstartvskip}{\vspace*{.5\baselineskip}}

%Ränder anzeigen
%\usepackage{showframe}

%Kapitel: Abstand Oben
% \RedeclareSectionCommand[%
%   beforeskip=0pt,
%   afterskip=1\baselineskip plus .1\baselineskip minus .167\baselineskip
% ]{chapter}

% Standard Packages

%nicht einrücken nach Absatz
% \setlength{\parindent}{0pt}

% Select what to do with command \comment:  
% \newcommand{\comment}[1]{}  %comment not showed
% \newcommand{\comment}[1]
% {\par {\bfseries \color{blue} #1 \par}} %comment showed

%### Todo Notes fuer Review von Betreuer #############
% Select what to do with todonotes: 
% \usepackage[disable]{todonotes} % notes not showed
\usepackage[draft]{todonotes}   % notes showed


%### Kopf- und Fußzeile ##############################
\pagestyle{fancy}

%Zeilenabstand -> singlespace, onehalfspace, doublespace
\usepackage{setspace}

%Kopfzeile mit Kapitel
\lhead{\slshape \leftmark}
\chead{}
\rhead{}
%%
\lfoot{}
\cfoot{\thepage}
\rfoot{}

%\renewcommand{\headrulewidth}{0pt}
\renewcommand{\footrulewidth}{0pt}


\newgeometry{left=2.5cm, right=2cm, top=2.5cm, bottom=3cm}

%1,5 Zeilenabstand
\onehalfspace
